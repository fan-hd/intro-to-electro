\chapter{Vector Analysis}

\section{Vector Algebra}
    \subsection{Vector Operations}
    \subsection{Vector Algebra: Component From}
    \subsection{Triple Products}
    \subsection{Position, Displacement, and Separation Vectors}
    \subsection{How Vectors Transform}

\section{Differential Calculus}
    \subsection{``Ordinary'' Derivatives}
    \subsection{Gradient}
    \subsection{The Del Operator}
    \subsection{The Divergence}
        \begin{prob}[1.15]
            Calculate the divergence of the following vector fuctions:

            (a) $\mathbf{v_a}=x^2\mathbf{\hat{x}} + 3xz^2\mathbf{\hat{y}} - 2xz\mathbf{\hat{z}}$.

            (b) $\mathbf{v_b} = xy\mathbf{\hat{x}} + 2yz\mathbf{\hat{y}} + 3zx \mathbf{\hat{z}}$.

            (c) $\mathbf{v_c} = y^2\mathbf{\hat{x}} + (2xy + z^2)\mathbf{\hat{y}} + 2yz\mathbf{\hat{z}}$.
        \end{prob}

        \begin{sol}[1.15]~\\
            (a)
            \begin{equation}
                \begin{aligned}
                    \nabla\cdot\mathbf{v_a} &=2x+0-2x \\
                    &= 0.
                \end{aligned}
            \end{equation}
            \noindent (b)
            \begin{equation}
                \begin{aligned}
                    \nabla\cdot\mathbf{v_b} &= y + 2z + 3x \\
                    &=3x + y + 2z.
                \end{aligned}
            \end{equation}
            \noindent (c)
            \begin{equation}
                \begin{aligned}
                    \nabla\cdot\mathbf{v_c} &= 0 + 2x + 2y \\
                    &=2x + 2y.
                \end{aligned}
            \end{equation}
        \end{sol}
        \begin{prob}[1.17]
            In two dimensions, 
            show that the divergence transforms as a scalar 
            under rotations.
        \end{prob}

        \begin{sol}[1.17]
        Since
            \begin{equation}
                \begin{pmatrix}
                    \overline{v}_y \\
                    \overline{v}_z \\
                \end{pmatrix}
        = \begin{pmatrix}
            \cos \theta & \sin \theta\\
            -\sin\theta & \cos \theta\\
        \end{pmatrix}
        \begin{pmatrix}
            v_y \\ v_z
        \end{pmatrix}
    \end{equation}

    and
    \begin{equation}
        \begin{pmatrix}
            \overline{y} \\ \overline{z}
        \end{pmatrix}
        = \begin{pmatrix}
            \cos \theta & \sin \theta \\
            -\sin \theta & \cos \theta
        \end{pmatrix}
        \begin{pmatrix}
            y \\ z
        \end{pmatrix},
    \end {equation}

    we have
        \begin{equation}
            \begin{aligned}
                &\parfpx{\overline{v}_x}{\overline{x}} + \parfpx{\overline{v}_y}{\overline{y}} + \parfpx{\overline{v}_z}{\overline{z}} \\
                =&\parfpx{v_x}{x} + \parfpx{\overline{v}_y}{y}\parfpx{y}{\overline{y}} + \parfpx{\overline{v}_y}{z}\parfpx{z}{\overline{y}}
                    + \parfpx{\overline{v}_z}{y}\parfpx{y}{\overline{z}} + \parfpx{\overline{v}_z}{z}\parfpx{z}{\overline{z}} \\
                =&\parfpx{v_x}{x} + \parfpx{\overline{v}_y}{y}\cos \theta 
                    + \parfpx{\overline{v}_y}{z}\sin \theta - \parfpx{\overline{v}_z}{y} \sin \theta + \parfpx{\overline{v}_z}{z} \cos \theta \\
                =&\parfpx{v_x}{x}+\parfpx{\left(\overline{v}_y\cos\theta - \overline{v}_z\sin\theta\right)}{y} + \parfpx{\left(\overline{v}_y\sin\theta + \overline{v}_z\cos\theta\right)}{z} \\
                =&\parfpx{v_x}{x}+\parfpx{v_y}{y} + \parfpx{v_z}{z}.
            \end{aligned}
        \end{equation}
\end{sol}

            \subsection{The Curl}
            \subsection{Product Rules}
            \begin{prob}[1.21]
                Prove product rules (i), (iv), and (v).
            \end{prob}
            \begin{sol}[1.21] ~\\
                (i):
                \begin{equation}
                    \begin{aligned}
                        \nabla(fg)&=\left(\parfpx{}{x}\mathbf{\hat{x}} + \parfpx{}{y}\mathbf{\hat{y}} + \parfpx{}{z}\mathbf{\hat{z}}\right)(fg) \\
                            &=\parfpx{(fg)}{x} + \parfpx{(fg)}{y} + \parfpx{(fg)}{z} \\
                            &=f\parfpx{}{x}g\mathbf{\hat{x}} + g\parfpx{}{x}f\mathbf{\hat{x}} + f\parfpx{}{y}g\mathbf{\hat{y}} 
                                + g\parfpx{}{y}f\mathbf{\hat{y}} + f\parfpx{}{z}g\mathbf{\hat{z}} + g\parfpx{}{z}f\mathbf{\hat{z}} \\
                            &=f\left(\parfpx{}{x}g\mathbf{\hat{x}} + \parfpx{}{y}g\mathbf{\hat{y}} + \parfpx{}{z}g\mathbf{\hat{z}}\right) 
                                + g\left(\parfpx{}{x}f\mathbf{\hat{x}} + \parfpx{}{y}f\mathbf{\hat{y}} + \parfpx{}{z}f\mathbf{\hat{z}}\right) \\
                            &=f \nabla g + g \nabla f.
                    \end{aligned}
                \end{equation}
            \end{sol}
            ~\\
                (iv):
                \begin{equation}
                    \begin{aligned}
                        \nabla\cdot(\mathbf{A}\times\mathbf{B}) &=\nabla\cdot(A_y B_z - A_z B_y, A_z B_x - A_x B_z, A_x B_y - A_y B_x) \\
                        &=A_y \parfpx{B_z}{x} - A_z \parfpx{B_y}{x} + A_z\parfpx{B_x}{y}\\
                        &\quad -A_x\parfpx{B_z}{y} +A_x\parfpx{B_y}{z} - A_y \parfpx{B_x}{z} \\
                        &=A_y\left(\parfpx{B_z}{x} - \parfpx{B_x}{z}\right) + A_x\left(\parfpx{B_y}{z} - \parfpx{B_z}{y}\right) \\
                        &\quad+ A_z\left(\parfpx{B_x}{y} - \parfpx{B_y}{x}\right) + B_x\left(\parfpx{A_z}{y} - \parfpx{A_y}{z}\right) \\
                        &\quad+ B_y\left(\parfpx{A_x}{z} - \parfpx{A_z}{x}\right) + B_z\left(\parfpx{A_y}{x} - \parfpx{A_x}{y}\right) \\
                        &=\mathbf{B}\cdot(\nabla\times\mathbf{A}) - \mathbf{A}\cdot(\nabla\times\mathbf{B})
                    \end{aligned}
                \end{equation}
                (v):
                \begin{equation}
                    \begin{aligned}
                        \nabla\times(f\mathbf{A}) &= \left(\parfpx{fA_z}{y} - \parfpx{fA_y}{z}, \parfpx{fA_x}{z} - \parfpx{fA_z}{x}, \parfpx{fA_y}{x} - \parfpx{fA_x}{y}\right) \\
                        &=f\nabla\times\mathbf{A} \\
                        &\quad + \left(A_z\parfpx{f}{y} - A_y\parfpx{f}{z}, A_x\parfpx{f}{z} - A_z\parfpx{f}{x}, A_y\parfpx{f}{x} - A_x\parfpx{f}{y}\right) \\
                        &=f\nabla\times\mathbf{A} + \left(\nabla f\right)\times\mathbf{A} \\
                        &=f\nabla\times\mathbf{A} - \mathbf{A}\times(\nabla f)
                    \end{aligned}
                \end{equation}

            \begin{prob}[1.22]~\\
                (a) If \textbf{A} and \textbf{B} are two vector functions, what does the expression $ (\mathbf{A}\cdot\nabla)\mathbf{B}$ mean?

                (That is, what are its $x$, $y$, and $z$ Components, in terms of the Cartesian Components of \textbf{A}, \textbf{B}, and $\nabla$?) \\

                \noindent (b) Compute $(\mathbf{\hat{r}} \cdot\nabla)\mathbf{\hat{r}}$, where $\mathbf{\hat{r}}$ is the unit vector defined in Eq.~1.21. \\

                \noindent (c) For the functions in Prob.~1.15, evaluate $(\mathbf{v_a}\cdot\nabla)\mathbf{v_b}$.

            \end{prob}

            \begin{sol}[1.22]~\\
                (a) Since 
                \begin{equation}
                    \mathbf{A} \cdot\nabla=A_x\parfpx{}{x} + A_y\parfpx{}{y} + A_z\parfpx{}{z},
                \end{equation}
                we have\\
                $x$ component: \\
                    \begin{equation}
                        A_x\parfpx{B_x}{x} + A_y\parfpx{B_x}{y} + A_z\parfpx{B_x}{z},
                    \end{equation}
                $y$ component:\\
                \begin{equation}
                    A_x\parfpx{B_y}{x} + A_y\parfpx{B_y}{y} + A_z\parfpx{B_y}{z},
                \end{equation}
                $z$ component:\\
                \begin{equation}
                    A_x\parfpx{B_z}{x} + A_y\parfpx{B_z}{x} + A_z\parfpx{B_z}{z}.
                \end{equation}\\
                \noindent (b) Since 
                \begin{equation}
                    \begin{aligned}
                        \mathbf{\hat{r}} &= \frac{\mathbf{r}}{r}\\
                        &=\frac{x\mathbf{\hat{x}} + y\mathbf{\hat{y}} + z\mathbf{\hat{z}}}{\sqrt{x^2 + y^2 + z^2}},
                    \end{aligned}
                \end{equation}
                and\\
                \begin{equation}
                    \begin{aligned}
                        \parfpx{(x/r)}{x} &= \frac{y^2+z^2}{r^3}, \\
                        \parfpx{(y/r)}{y} &= \frac{x^2+z^2}{r^3}, \\
                        \parfpx{(z/r)}{z} &= \frac{x^2+y^2}{r^3},
                    \end{aligned}
                \end{equation}
                    we have
                \begin{equation}
                    \begin{aligned}
                        \parfpx{\mathbf{\hat{r}}}{x} &= \frac{r^2\mathbf{\hat{x}}-\mathbf{r}\cdot x}{r^3}, \\
                        \parfpx{\mathbf{\hat{r}}}{y} &= \frac{r^2\mathbf{\hat{y}}-\mathbf{r}\cdot y}{r^3}, \\
                        \parfpx{\mathbf{\hat{r}}}{z} &= \frac{r^2\mathbf{\hat{z}}-\mathbf{r}\cdot z}{r^3}.
                    \end{aligned}
                \end{equation}
                    Thus,
                \begin{equation}
                    \begin{aligned}
                    (\mathbf{\hat{r}} \cdot \nabla)\mathbf{\hat{r}} &= \frac{x}{r} \parfpx{\mathbf{\hat{r}}}{x} + \frac{y}{r} \parfpx{\mathbf{\hat{r}}}{y} + \frac{z}{r}\parfpx{\mathbf{\hat{r}}}{z}\\
                        &=0.
                    \end{aligned}
                \end{equation}

                \noindent (c) Since
                \begin{equation}
                    \begin{aligned}
                        \parfpx{\mathbf{v_b}}{x} &=y\mathbf{\hat{x}} + 3z\mathbf{\hat{z}}, \\
                        \parfpx{\mathbf{v_b}}{y} &=x\mathbf{\hat{x}} + 2z\mathbf{\hat{y}}, \\
                        \parfpx{\mathbf{v_b}}{z} &=2y\mathbf{\hat{y}} + 3x \mathbf{\hat{z}},
                    \end{aligned}
                \end{equation}
                \begin{equation}
                    \begin{aligned}
                            (\mathbf{v_a\cdot\nabla})\mathbf{v_b} &= x^2(y\mathbf{\hat{x}} + 3z\mathbf{\hat{z}})
                                +3xz^2(x\mathbf{\hat{x}} + 2z\mathbf{\hat{y}})
                                -2xz(2y\mathbf{\hat{y}} + 3x\mathbf{\hat{z}}) \\
                            &=(x^2y + 3x^2z^2)\mathbf{\hat{x}} + (6xz^3 - 4xyz)\mathbf{\hat{y}} -3x^2z\mathbf{\hat{z}}.
                    \end{aligned}
                \end{equation}
            \end{sol}

            \begin{prob}[1.23](For masochists only.) Prove products rules (ii) and (vi). Refer to Prob.~1.22 for the definition of $(\mathbf{A}\cdot\nabla)\mathbf{B}$.
            \end{prob}

            \begin{prob}[1.24] Derive the three quotient rules.
            \end{prob}

            \begin{sol}[1.24]
                \begin{equation}
                    \begin{aligned}
                        \nabla\left(\frac{f}{g}\right)&=f\nabla\left(\frac{1}{g}\right) + \frac{1}{g}\nabla f \\
                        &=f\cdot\left(-\frac{\nabla g}{g^2}\right) + \frac{\nabla f}{g} \\
                        &=\frac{g\nabla f - f\nabla g}{g^2}.
                    \end{aligned}
                \end{equation}
                \begin{equation}
                    \begin{aligned}
                        \nabla\left(\frac{\mathbf{A}}{g}\right) &= \frac{\nabla \cdot A}{g} + \mathbf{A}\cdot \nabla \left(\frac{1}{g}\right) \\
                        &=\frac{g\left(\nabla\cdot\mathbf{A}\right)-\mathbf{A}\cdot\nabla g}{g^2}.
                    \end{aligned}
                \end{equation}~\\
                \begin{equation}
                    \begin{aligned}
                        \nabla\times\left(\frac{\mathbf{A}}{g}\right) &= \frac{1}{g} \nabla\times\mathbf{A} - \mathbf{A} \times \nabla\left(\frac{1}{g}\right) \\
                        &=\frac{g(\nabla\times\mathbf{A}) + \mathbf{A}\times(\nabla g)}{g^2}.
                    \end{aligned}
                \end{equation}
            \end{sol}
    \subsection{Second Derivatives}
        \begin{prob}[1.26] Calculate the Laplacian of the following functions:

            \noindent (a) $T_a = x^2 + 2xy + 3z + 4$.
            
            \noindent (b) $T_b = \sin x \sin y \sin z$.

            \noindent (c) $e^{-5x} \sin 4y \cos 3z$.

            \noindent (d) $\mathbf{v} = x^2\mathbf{\hat{x}} + 3xz^2\mathbf{\hat{y}} - 2xz\mathbf{\hat{z}}$.
        \end{prob}

        \begin{sol}[1.26]~\\
            (a) $\nabla^2 T_a=2$.

            \noindent (b) $\nabla^2 T_b = -3\sin x \sin y\sin z$.

            \noindent (c) $\nabla^2 T_c = 0$.

            \noindent (d) $\nabla^2 \mathbf{v}=2\mathbf{\hat{x}} + 6x\mathbf{\hat{y}}$.
        \end{sol}

        \begin{prob}[1.27] Prove that the divergence of a curl is always zero. \textit{Check} it for function $\mathbf{v_a}$ in Prob.~1.15.
        \end{prob}

        \begin{sol}[1.27]
            \begin{equation}
                \begin{aligned}
                    \nabla\cdot(\nabla\times\mathbf{v}) &= \nabla\cdot\left(\left(\parfpx{v_z}{y} - \parfpx{v_y}{z}\right)\mathbf{\hat{x}} + \left(\parfpx{v_x}{z} - \parfpx{v_z}{x}\right)\mathbf{\hat{y}} + \left(\parfpx{v_y}{x} - \parfpx{v_x}{y}\right)\mathbf{\hat{z}}\right) \\
                    &=\parfpx{}{x}\parfpx{v_z}{y} - \parfpx{}{x}\parfpx{v_y}{z} + \parfpx{}{y}\parfpx{v_x}{z} - \parfpx{}{y}\parfpx{v_z}{x}\\ 
                    &\quad + \parfpx{}{z}\parfpx{v_y}{x} - \parfpx{}{z}\parfpx{v_x}{y} \\
                    &=\left(\frac{\partial^2v_z}{\partial x\partial y} -\frac{\partial^2v_z}{\partial y\partial x}\right) + \left(\frac{\partial^2v_x}{\partial y\partial z} - \frac{\partial^2v_x}{\partial z\partial y}\right) \\
                    &\quad + \left(\frac\partial{v_y}{\partial z\partial x} - \frac{\partial^2v_y}{\partial x\partial z}\right)\\
                    &=0.
                \end{aligned}
            \end{equation}
        \end{sol}

        \begin{prob}[1.28]Prove that the curl of a gradient is always zero. \textit{Check} it for function (b) in Prob.~1.11.
        \end{prob}

        \begin{sol}[1.28]
            \begin{equation}
                \begin{aligned}
                    \nabla\times(\nabla f) &= \nabla\times\left(\parfpx{f}{x}\mathbf{\hat{x}}+\parfpx{f}{y}\mathbf{\hat{y}} + \parfpx{f}{z}\mathbf{\hat{z}}\right) \\
                    &=\left(\frac{\partial^2f}{\partial y\partial z} - \frac{\partial^2 f}{\partial z \partial y}\right)\mathbf{\hat{x}} + \left(\frac{\partial^2f}{\partial z \partial x}- \frac{\partial^2f}{\partial x \partial z}\right)\mathbf{\hat{y}} \\
                    &\quad + \left(\frac{\partial^2f}{\partial x \partial y} - \frac{\partial^2 f}{\partial y \partial x}\right)\mathbf{\hat{z}} \\
                    &=0.
                \end{aligned}
            \end{equation}
        \end{sol}

\section{INTEGRAL CALCULUS}

\subsection{Line, Surface, and Volume Integrals}
\begin{prob}[1.29] Calculate the line integral of the function $\mathbf{v}=x^2\mathbf{\hat{x}} + 2yz\mathbf{\hat{y}} + y^2\mathbf{\hat{z}}$ from the origin to the point (1, 1, 1) by three different routes:

    \noindent (a) (0, 0, 0) $\rightarrow$(1, 0, 0) $\rightarrow$(1, 1, 0) $\rightarrow$(1, 1, 1).

    \noindent (b) (0, 0, 0) $\rightarrow$(0, 0, 1) $\rightarrow$(0, 1, 1)$\rightarrow$(1, 1, 1).

    \noindent (c) The direct straight line.

    \noindent (d) What is the line integral around the closed loop that goes \textit{out} along path (a) and \textit{back} along path (b)?
\end{prob}

\begin{sol}[1.29]~\\
    \noindent (a) $d\mathbf{l} = dx\mathbf{\hat{x}} + dy\mathbf{\hat{y}} + dz\mathbf{\hat{z}}$.

    (i) $d\mathbf{l} = dx \mathbf{\hat{x}}$,$y=0$,$z=0$,so
    \begin{equation}
        \int \mathbf{v}\cdot d\mathbf{l} = \int^1_0 x^2 dx = \frac{1}{3}.
    \end{equation}

    (ii) $d\mathbf{l} = dy \mathbf{\hat{y}}$,$x=1$,$z=0$, so
    \begin{equation}
        \int \mathbf{v}\cdot d\mathbf{l} = \int_0^10 fy = 0.
    \end{equation}

    (iii) $d\mathbf{l} = dz\mathbf{\hat{z}}$,$x=1$,$y=1$, so
    \begin{equation}
        \int \mathbf{v}\cdot d\mathbf{l} = \int_0^11dz = 1.
    \end{equation}

    Then,
    \begin{equation}
        \int_{L_a} \mathbf{v}\cdot d\mathbf{l} = \frac{1}{3} + 0 + 1 = \frac{4}{3}.
    \end{equation}

    \noindent (b) Similarly,

    (i) $d\mathbf{l} = dz \mathbf{\hat{z}}$, $x = 0$, $y = 0$, so
    \begin{equation}
        \int \mathbf{v} \cdot\mathbf{l} = \int_0^1 0dz = 0.
    \end{equation}

    (ii) $d\mathbf{l} = dy\mathbf{\hat{y}}$, $x = 0$, $z = 1$, so
    \begin{equation}
        \int \mathbf{v} \cdot\mathbf{l} = \int_0^1 2ydy = 1.
    \end{equation}

    (iii) $d\mathbf{l} = dx\mathbf{\hat{x}}$, $y = z = 1$, so
    \begin{equation}
        \int\mathbf{v}\cdot\mathbf{l} = \int_0^1x^2dx = \frac{1}{3}.
    \end{equation}

    Thus,
    \begin{equation}
        \int_{L_b}\mathbf{v}\cdot\mathbf{l} = 0 + 1 + \frac{1}{3} = \frac{4}{3}.
    \end{equation}

    \noindent (c) Since $x = y = z$ along the direct straight line, we have $d\mathbf{l} = dx\mathbf{\hat{x}} + dy\mathbf{\hat{y}} + dz\mathbf{\hat{z}} = dx(\mathbf{\hat{x}} + \mathbf{\hat{y}} + \mathbf{\hat{z}})$. Thus,
    \begin{equation}
        \int_L \mathbf{v}\cdot\mathbf{l} = \int_0^14x^2dx = \frac{4}{3}.
    \end{equation}

    \noindent (d)
    \begin{equation}
        \oint\mathbf{v}\cdot\mathbf{l} = \int_{La}\mathbf{v}\cdot\mathbf{l} - \int_{L_b}\mathbf{v}\cdot\mathbf{l} = 0.
    \end{equation}
\end{sol}

\begin{prob}[1.30] Calculate the surface integral of the function in Ex.~1.7, over the \textit{bottom} of the box. For consistency, let ``upward'' be the positive direction. Does the surface integral depend only on the boundary line for this function? What is the total flux over the \textit{closed} surface of the box ( \textit{including} the bottom)?[ \textit{Note}: For the \textit{closed} surface, the positive direction is ``outward,'' and hence ``down,'' for the bottom face.]
\end{prob}

\begin{sol}[1.30] For the bottom side, $z = 0$, $d\mathbf{a} = dxdy\mathbf{\hat{z}}$, $\mathbf{v}\cdot\mathbf{a} = -3ydxdy$, so
    \begin{equation}
        \int\mathbf{v}\cdot\mathbf{a} = -3\int_0^2dx\int_0^2ydy = -12.
    \end{equation}

    Thus the surface integral of the function over the bottom of the box is -12, and the surface integral doesn't depend only on the boundary line for this function.

    The total flux over the closed surface of the box is
    \begin{equation}
        \oiint_S \mathbf{v}\cdot\mathbf{a} = 20 + 12 = 32.
    \end{equation}
\end{sol}

\begin{prob}[1.31] Calculate the volume integral of the function $T = z^2$ over the terahedron with corners at (0,0,0), (1,0,0), (0,1,0), and (0,0,1).
\end{prob}

\begin{sol}[1.31]
    \begin{equation}
        \iiint_V T dx dy dz = \iiint_V z^2 dx dy dz = \int_0^1\frac{(1-z)^2}{2}z^2dz = \frac{1}{6} - \frac{1}{4} + \frac{1}{10} = \frac{1}{60}.
    \end{equation}
\end{sol}
